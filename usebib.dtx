% \iffalse meta-comment
%
% Copyright (C) 2012 by Enrico Gregorio 
% <Enrico dot Gregorio at univr dot it>
% -------------------------------------------------------
% 
% This work may be distributed and/or modified under the
% conditions of the LaTeX Project Public License, either
% version 1.3c of this license or (at your option) any
% later version. The latest version of this license is in
%   http://www.latex-project.org/lppl.txt
% and version 1.3 or later is part of all distributions
% of LaTeX version 2005/12/01 or later.
%
% This work has the LPPL maintenance status `maintained'.
% 
% The Current Maintainer of this work is Enrico Gregorio.
%
% This work consists of the files
%   usebib.dtx 
%   usebib.ins
% and the derived file usebib.sty.
%
% \fi
%
% \iffalse
%<*driver>
\ProvidesFile{usebib.dtx}
%</driver>
%<package>\NeedsTeXFormat{LaTeX2e}[2008/04/05]
%<package>\ProvidesPackage{usebib}
%<*package>
    [2012/04/13 v1.0a Reusing bib files]
%</package>
%
%<*driver>
\documentclass{ltxdoc}
\providecommand{\BibTeX}{\textsc{Bib}\!\TeX}
\EnableCrossrefs
\CodelineIndex
\RecordChanges
\begin{document}
  \DocInput{usebib.dtx}
  \PrintChanges
  \PrintIndex
\end{document}
%</driver>
% \fi
%
% \CheckSum{128}
%
% \CharacterTable
%  {Upper-case    \A\B\C\D\E\F\G\H\I\J\K\L\M\N\O\P\Q\R\S\T\U\V\W\X\Y\Z
%   Lower-case    \a\b\c\d\e\f\g\h\i\j\k\l\m\n\o\p\q\r\s\t\u\v\w\x\y\z
%   Digits        \0\1\2\3\4\5\6\7\8\9
%   Exclamation   \!     Double quote  \"     Hash (number) \#
%   Dollar        \$     Percent       \%     Ampersand     \&
%   Acute accent  \'     Left paren    \(     Right paren   \)
%   Asterisk      \*     Plus          \+     Comma         \,
%   Minus         \-     Point         \.     Solidus       \/
%   Colon         \:     Semicolon     \;     Less than     \<
%   Equals        \=     Greater than  \>     Question mark \?
%   Commercial at \@     Left bracket  \[     Backslash     \\
%   Right bracket \]     Circumflex    \^     Underscore    \_
%   Grave accent  \`     Left brace    \{     Vertical bar  \|
%   Right brace   \}     Tilde         \~}
%
%
% \changes{v1.0a}{2012/04/13}{Corrected macro names in the doc}
% \changes{v1.0}{2012/03/17}{Initial version}
%
% \GetFileInfo{usebib.dtx}
%
% \DoNotIndex{\newcommand,\newenvironment,\!,\@empty,\@gobble,\@gobbletwo}
% \DoNotIndex{\@ifpackageloaded,\@ifpackagewith,\@ifundefined,\@namedef}
% \DoNotIndex{\@nil,\@onlypreamble,\@tempa,\@tempb,\@tempswafalse,\def}
% \DoNotIndex{\@tempswatrue,\^,\-,\active,\begingroup,\catcode,\@car,\@cdr}
% \DoNotIndex{\edef,\else,\endgroup,\endinput,\expandafter,\fi,\if}
% \DoNotIndex{\if@tempswa,\ifcase,\ifnum,\ifx,\lccode,\let,\lowercase}
% \DoNotIndex{\MessageBreak,\next,\number,\numexpr,\or,\PackageError}
% \DoNotIndex{\PackageWarning,\PackageWarningNoLine,\strip@prefix,\@@end}
% \DoNotIndex{\relax,\space,\string,\DeclareOption,\ProcessOptions}
% \DoNotIndex{\meaning,\ifdefined,\csname,\chardef,\endcsname,\protect}
% \DoNotIndex{\input,\RequirePackage,\global,\ifcsname,\makeatother}
% \DoNotIndex{\@makeother,\the,\toks@}
%
% \title{The \textsf{usebib} package\thanks{This document
%   corresponds to \textsf{newunicodechar}~\fileversion, dated \filedate.}}
% \author{Enrico Gregorio \\ \texttt{Enrico dot Gregorio at univr dot it}}
%
% \maketitle
%
% \section{Introduction}
%
% With \textsf{biblatex} it's possible to use any element of a
% bibliographic item in a document. With packages such as
% \textsf{natbib} or simply with the native \BibTeX{} support one
% can't refer directly to the title of a cited article or book without
% directly typing it in.
%
% This package allows this by reading in a special environment a
% |.bib| file. It should be considered no more than a devious usage of
% a |.bib| file, to be honest. But it works.
%
% \section{Usage}
%
% The package \emph{must} be loaded after \textsf{hyperref}, but it
% doesn't require it. It depends only on \textsf{url} and
% \textsf{keyval}, which should be present on every \TeX{}
% distribution.
%
% \subsection{Preamble commands}
% 
% \DescribeMacro{\bibinput}
% The main command is |\bibinput|, which accepts the name of a |.bib|
% file (\emph{without} the extension) and may be repeated any number
% of times to read all the necessary files.\footnote{Making it
% dependent on the \texttt{\string\bibliography} command would be too
% difficult.}  The command should appear only in the preamble.
%
% \DescribeMacro{\newbibfield}
% With the macro |\newbibfield|, to be issued \emph{before}
% |\bibinput|, it's possible to declare new known fields, for example
%\begin{verbatim}
% \newbibfield{annotation}
%\end{verbatim}
% and say in the document |\usebibentry{xyz}{annotation}|.  It's also
% possible to say |\newbibfield{author}|, but as author names are
% input in a very peculiar way, the result would be at least
% questionable in the vast majority of cases. This command should
% appear before |\bibinput|.
%
% \DescribeMacro{\newbibignore}
% The entry types |@preamble| and |@string| are ignored by default. If
% some other type should also, it's possible to say
% \begin{flushleft}
% |\newbibignore{|\meta{type}|}|
% \end{flushleft}
% so that, for example, |\newbibignore{misc}| will ignore all |@misc|
% entry types. This command should appear before |\bibinput|.
% 
% \subsection{Document commands}
%
% \DescribeMacro{\usebibentry}
% The main document command is |\usebibentry| which requires two
% arguments, the bibliography key and the field name:
% \begin{flushleft}
% |\usebibentry{|\meta{key}|}{|\meta{field}|}|
% \end{flushleft}
% The \meta{key} is what's used in the argument to |\cite| (or
% variations thereof) and \meta{field} can be |title|, |year|, |note|
% or |isbn|, but it's possible to augment the list of known
% fields. So, in order to get in the document the referenced item's
% title one can say
%\begin{verbatim}
% In the book ``\usebibentry{newton1687}{title}''~\cite{newton1687},
% published in~\usebibentry{newton1687}{year}, Newton
% describes the Law of Universal Gravitation
%\end{verbatim}
% to get the text
% \begin{quote}
% In the book ``Philosophiae Naturalis Principia Mathematica''~[3],
% published in 1687, Newton describes the Law of Universal Gravitation
% \end{quote}
% assuming that in the |.bib| file that has been read with |\bibinput|
% there's an entry such as
%\begin{verbatim}
% @book{newton1687,
%  author={Newton, Isaac},
%  title={Philosophiae Naturalis Principia Mathematica},
%  publisher={Royal Society},
%  address={London},
%  year={1687},
%  url={http://royalsociety.org},
% }
%\end{verbatim}
% Unknown keys and missing values trigger errors: so
% |\usebibentry{newton1687}{isbn}| would result in the error
%\begin{verbatim}
% Undefined key `newton1687' or empty value for `isbn'
%\end{verbatim}
% Unsupported fields will be ignored and print nothing.
%
% \DescribeMacro{\usebibentryurl}
% If the |.bib| entry has a |url| field, its value can be obtained
% with
% \begin{flushleft}
% |\usebibentryurl{|\meta{key}|}|
% \end{flushleft}
% \begin{trivlist}\footnotesize\item
% When \textsf{hyperref} is not loaded, this is equivalent to calling
% \verb+\url|+\meta{url}\verb+|+. In case the URL contains the
% character \verb+|+, this call will produce wrong results and
% one can specify a different delimiter
% \begin{flushleft}
% |\usebibentry[!]{|\meta{key}|}|
% \end{flushleft}
% where, of course, the optional argument is a character not appearing
% in the URL. This optional argument is not needed when
% \textsf{hyperref} is loaded and, if present, it's ignored.
% \end{trivlist}
%
% \section{Limitations}
%
% The field delimiters in the |.bib| files \emph{must} be braces, as
% must also the entry delimiters. \BibTeX{} allows double quotes in
% for delimiting fields and parentheses for the entry. So, while
%\begin{verbatim}
% @book(newton1687,
%  author="Newton, Isaac",
%  title="Philosophiae Naturalis Principia Mathematica",
%  publisher="Royal Society",
%  address="London",
%  year=1687,
%  url="http://royalsociety.org",
% )
%\end{verbatim}
% would be accepted by \BibTeX{}, it won't be by
% \textsf{usebib}. Supporting this is out of question, since we parse
% entries as a key-value list. Strictly speaking, double quotes won't
% give errors, as long as no comma is in the fields, but will be part
% of the replacement text anyways. Numeric field values can, of
% course, be input without field delimiters.
%
% Another limitation is in the fact that field names should always be
% lowercase; supporting mixed case for these is beyond the limits of a
% small package like this. Sorry.
%
% Note also that all the bibliographic data are loaded in memory, so
% large |.bib| files could exceed \TeX{} capacity.
%
% Don't ever try typing |\usebibentry{foo}{url}|: you've been
% warned. There's |\usebibentryurl{foo}| for this.
%
% Finally, the package requires $\varepsilon$-\TeX, which shouldn't be
% a problem, nowadays.
%
% \StopEventually{}
%
% \section{Implementation}
%
% After the usual presentation, we check first of all that the
% typesetting engine is sufficiently recent to include
% $\varepsilon$-\TeX{} extensions.
%    \begin{macrocode}
\@ifundefined{eTeXversion}
  {\PackageError{usebib}{LaTeX engine too old, aborting}
   {Please upgrade your TeX system}\@@end}{}
%    \end{macrocode}
% We start by loading the required packages.
%    \begin{macrocode}
\RequirePackage{url}
\RequirePackage{keyval}
%    \end{macrocode}
% 
% When we read a |.bib| file the |@| character is made active. Its
% action is to read what follows up to the next open brace, using it
% as the value of the |currenttype| key. This is necessary for
% ignoring the entry types |@preamble| and |@string| which would mess
% up everything.
%
% Now we define the known fields as keys for \textsf{keyval} and the
% two auxiliary keys. The |\define@reuse@key| macro is just syntactic
% sugar.
%    \begin{macrocode}
\define@key{usebib}{currenttype}{\def\@tempa{#1}}
\define@key{usebib}{currententry}{\def\reuse@current{#1}}
\def\define@reuse@key#1{%
  \define@key{usebib}{#1}{\global\@namedef{reuse@\reuse@current @#1}{##1}}}
\define@reuse@key{title}
\define@reuse@key{pages}
\define@reuse@key{isbn}
\define@reuse@key{url}
\define@reuse@key{year}
\define@reuse@key{note}
%    \end{macrocode}
%
% \begin{macro}{\reuse@find}
% The first macro to be used when looking at an entry, later we'll
% define the active |@| to be this macro: it takes as argument the
% entry's type and does, for instance,
% |\setkeys{usebib}{currenttype=BOOK}|, but inside a |\lowercase| so
% that the type will be normalized. Then it checks whether
% |\reuse@type@book| is defined, which it isn't, so |\reuse@extract|
% will be expanded; since we later define |\reuse@type@preamble| and
% |\reuse@type@string|, such entries will be ignored because of
% |\@gobble|. The expansion of |\reuse@extract| is done in a group
% where |@| has category code~12 and also the percent character,
% because it can be in some URL (and has not a special meaning in
% \BibTeX{} entries).
%    \begin{macrocode}
\def\reuse@find#1#{%
  \lowercase{\setkeys{usebib}{currenttype=#1}}%
  \ifcsname reuse@type@\@tempa\endcsname
    \expandafter\@gobble
  \else
    \begingroup\@makeother\%\makeatother
    \expandafter\reuse@extract
  \fi}
%    \end{macrocode}
% \end{macro}
%
% \begin{macro}{\reuse@extract}
% The entire entry is the argument to |\reuse@extract|; since it has a
% fixed form, we feed the citation key as the value of |currententry|
% and then read all the fields as key-value pairs. The effect of
% |currententry=xyz| is to define |\reuse@current| to |xyz|; so a
% subsequent |year=2012| will do the equivalent of 
%\begin{verbatim}
% \gdef\reuse@xyz@year{2012}
%\end{verbatim}
% and similarly for the other known fields. The group opened by
% |\reuse@find| is then closed.
%    \begin{macrocode}
\def\reuse@extract#1{\setkeys{usebib}{currententry=#1}\endgroup}
%    \end{macrocode}
% \end{macro}
%
% \begin{macro}{\reuse@type@preamble}
% \begin{macro}{\reuse@type@preamble}
% \begin{macro}{\newbibignore}
% Here we define the two entry types to ignore and the user level
% command to augment the list of ignored fields.
%    \begin{macrocode}
\let\reuse@type@preamble\@empty
\let\reuse@type@string\@empty
\def\newbibignore#1{\expandafter\let\csname reuse@type@#1\endcsname\@empty}
%    \end{macrocode}
% \end{macro}
% \end{macro}
% \end{macro}
%
% \begin{macro}{\reuse@error}
% A macro for issuing an error message.
%    \begin{macrocode}
\def\reuse@error#1#2{%
  \PackageError{usebib}
    {Undefined key `#1' or empty value for `#2'}
    {The key you used is wrong or the value to `#2' has not been set}}
%    \end{macrocode}
% \end{macro}
%
% Next we define the other four user level macros.
% \begin{macro}{\newbibfield}
% |\newbibfield{type}| defines a new key.
%    \begin{macrocode}
\newcommand\newbibfield[1]{\define@reuse@key{#1}}
%    \end{macrocode}
% \end{macro}
%
% \begin{macro}{\usebibentry}
% \begin{macro}{\usebibentryurl}
% \begin{macro}{\reuse@usebibentryurl}
% Both |\usebibentry| and |\usebibentryurl| check whether the
% corresponding inner macro is defined. The latter then expands a
% special inner version that's defined in different ways depending
% whether \textsf{hyperref} has been loaded or not. If it is, a simple
% |\scantokens| hands the job to the |\url| macro, which is cleverly
% defined for accepting special characters; otherwise we need to be
% sure that the URL is properly delimited.
%    \begin{macrocode}
\def\usebibentry#1#2{\@ifundefined{reuse@#1@#2}
  {\reuse@error{#1}{#2}}
  {\@nameuse{reuse@#1@#2}}}
\newcommand{\usebibentryurl}[2][|]{\@ifundefined{reuse@#2@url}
  {\reuse@error{#2}{url}}
  {\reuse@usebibentryurl{#1}{#2}}}
\@ifpackageloaded{hyperref}{\@tempswatrue}{\@tempswafalse}
\if@tempswa
  \def\reuse@usebibentryurl#1#2{%
    \scantokens{\url{\csname reuse@#2@url\endcsname}\endinput}}
\else
  \def\reuse@usebibentryurl#1#2{%
    \toks@=\expandafter\expandafter\expandafter
      {\csname reuse@#2@url\endcsname}%
    \scantokens\expandafter{%
      \expandafter\url\expandafter#1\the\toks@#1\endinput}}
\fi
%    \end{macrocode}
% \end{macro}
% \end{macro}
% \end{macro}
%
% \begin{macro}{\bibinput}
% The macro that starts the reading of a |.bib| file. We start a group
% where end-of-line characters are treated like spaces and where the
% error messages of \textsf{keyval} and \textsf{xkeyval} about
% undefined keys are disabled. Then we define the active |@| to be
% |\reuse@find| and input the |.bib| file, closing the group at the
% end.
%    \begin{macrocode}
\newcommand{\bibinput}[1]{%
  \begingroup
  \catcode`\^^M=10
  \let\KV@err=\@gobble
  \let\KV@errx=\@gobble
  \let\XKV@err=\@gobble
  \begingroup\lccode`\~=`\@
    \lowercase{\endgroup\let~}\reuse@find
  \catcode`\@=\active \input{#1.bib}\endgroup}
%    \end{macrocode}
% \end{macro}
% \Finale
\endinput
